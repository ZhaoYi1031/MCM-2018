\section{Introduction}
\subsection{Problem Background}
Here is the problem background ...

Two major problems are discussed in this paper, which are:
\begin{itemize}
    \item Doing the first thing.
    \item Doing the second thing.
\end{itemize}

\subsection{Literature Review}
A literatrue\cite{1} say something about this problem ...

\subsection{Our work}
We do such things ...

\begin{enumerate}[\bfseries 1.]
    \item We do ...
    \item We do ...
    \item We do ...
\end{enumerate}

\section{Preparation of the Models}
\subsection{Assumptions}

\subsection{Notations}
The primary notations used in this paper are listed in \textbf{Table \ref{Ntt}}.
\begin{table}[h]
    \begin{center}
        \caption{Notations}
        \begin{tabular}{cl}
            \toprule
            \multicolumn{1}{m{3cm}}{\centering Symbol}
            &\multicolumn{1}{m{8cm}}
              {\centering Definition}\\
            \midrule
            $A$&the first one\\
            $b$&the second one\\
            $\alpha$ &the last one\\
            \bottomrule
        \end{tabular}\label{Ntt}
    \end{center}
\end{table}

\section{The Models}

\subsection{Model 1}
\subsubsection{Detail 1 about Model 1}
\begin{equation}
    e^{i\theta}=\cos\theta+i\sin\theta.
\end{equation}

\section{Strengths and Weaknesses}
\subsection{Strengths}
\begin{itemize}
    \item First one ...
    \item Second one ...
\end{itemize}

\subsection{Weaknesses}
\begin{itemize}
    \item Only one ...
 \end{itemize}

\begin{thebibliography}{99}
\addcontentsline{toc}{section}{Refenrence}  %引用部分标题("Refenrence")的重命名
\bibitem{1}Elisa T. Lee, Oscar T. Survival Analysis in Public Health Research. \emph{Go.College of Public Health}, 1997(18):105-134.
\bibitem{2}Wikipedia: Proportional hazards model. 2017.11.26. \texttt{\\https://en.wikipedia.org/wiki/Proportional\_{}hazards\_{}model}
\end{thebibliography}

